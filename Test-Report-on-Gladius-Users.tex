\documentclass[12pt]{article}
\usepackage[utf8]{inputenc}
\usepackage{graphicx}
\usepackage{float}
\graphicspath{ {diagrams/} }


\title{ \includegraphics[width=10cm]{up} \\ [0.5cm] COS 301 2017 - Round 4\\ Broadsword Users\\ Testing of Gladios Users }

\author{Monkeli Dilapisho \hfill 15074260 \\ Drew Langley \hfill 11039753 \\ Dawie Pritchard \hfill [Student Number] \\ Cameron Trivella \hfill 14070970}
\date{5 May 2017}


\begin{document}
\maketitle
\pagebreak
\tableofcontents
\pagebreak


\section{Abstract}
	For round 4 of the COS 301 class project of 2017, in the wake of the implementation phase, the Broadsword Users team has been tasked with testing the Gladios Users team's 		implementation. We created a test model based on the high level functional requirements of the NavUP Software Requirements Specification document, so as to test the various 		specifications of the implementation, covering the core functions and innovations implemented.
	\\ 
	Testing was performed on the following service contracts and functions:
	\begin{itemize}
		\item Register
		\item Login
		\item Remove
		\item Get Details
		\item Get Email
		\item Get Phone 
		\item Get User
		\item Activate User
	\end{itemize}	

\pagebreak
	
\section{Service Contracts}
Tests were done using scripts and driver programs to ensure the implementation was tested as if it were operational in the real world.	
	\noindent The testing will consisted of a check of the following conditions for each or the services with a large focus on the retrieval of a user's data:\\
	 	
	 	\begin{itemize}
		\item Precondition: a condition that must be satisfied for the function to perform correctly.
		\item Exception: an exception that is thrown if the precondition is not met.
		\item Postcondition: a change seen in the system after successful invocation of the function.
		\item Return: the value returned by the function in query.
		\end{itemize}
			
\subsection{Register, Login and Removal of users}
	The functions to allow users to CRUD their personal data were up to standard and performed well under all tests.
	
	\begin{itemize}
		\item \textbf{registerUser - }The function recieves a user's username, full name, email, password and phone number as parameters and persists this information to the 				database so as to register the user. \textbf{Mark: 10/10}
		\item\textbf{login - }The function recieves username and password as parameters and check if the user has been registered in the database.\textbf{Mark: 10/10}
		\item \textbf{removeUser - } The function recieves a username as a parameter and delets the users information from the database.\textbf{Mark: 10/10}
		\end{itemize}
\pagebreak	
\subsection{Retrieval of User data}
	Multiple function swere implemented to allow for the retrieval of user data, namely: getUserDetails(). getEmail() and getPhoneNumber()
	\indent *note getUser(username) is presumed to have be renamed getUserDetails(username). getEmail(username) and getPhoneNumber(username). are also available.
	
	\subsubsection{getUserDetails}
		This function recieves a username as a string, and checks if the username exists in the database. Returns a semi colon separated user object.
	
	\begin{itemize}
		\item Precondition: userName is a registered user
		\item Exception: If userName is not a registered user throw noSuchUser exception
		\item Postcondition: No change
		\item Return: a userObject
	\end{itemize}
		
	\noindent \textbf{Test:} getUserDetails() called on existing user.\\
	\textbf{Result:} Returned string containing Fullname, email, phone number.\\\\
	\textbf{Test:} getUserDetails() called on non-existing user.\\
	\textbf{Result:} Returned string "the user does not exist".\\
	
	%Test: getEmail(username) called on existing user.\\
	%Result: Returned string containing the correct email.\\\\
	%Test: getPhoneNumber(username) called on existing user.\\
	%Result: Returned string containing the correct phone number.\\
	
	\noindent Note: Although no getUser(username) is available, nor any function which returns an object containing the details of the user, restrictions placed on the group by the 			integration team may have made these options preferable.\\ \\ \textbf{Mark: 8/10}
	
	
	\subsubsection{getEmail}
		This function recieves a username as a string, and checks if the username exists in the database. Returns an email string.
	
	\begin{itemize}
		\item Precondition: userName is a registered user
		\item Exception: If userName is not a registered user throw noSuchUser exception
		\item Postcondition: No change
		\item Return: a user's email
	\end{itemize}
		
	\noindent \textbf{Test:} getEmail() called on existing user.\\
	\textbf{Result:} Returned string the user's email.\\\\
	\textbf{Test:} geEmail() called on non-existing user.\\
	\textbf{Result:} Returned string "the user does not exist".\\
	\\ \\
	\textbf{Mark: 8/10}
	
	
	\subsubsection{getPhoneNumber}
		This function recieves a username as a string, and checks if the username exists in the database. Returns a phone number string.
	
	\begin{itemize}
		\item Precondition: userName is a registered user.
		\item Exception: If userName is not a registered user throw noSuchUser exception.
		\item Postcondition: No change.
		\item Return: a phone number.
	\end{itemize}
		
	\noindent \textbf{Test:} getPhoneNumber() called on existing user.\\
	\textbf{Result:} Returned string containing the user's phone number.\\\\
	\textbf{Test:} getPhoneNumber() called on non-existing user.\\
	\textbf{Result:} Returned string "the user does not exist".\\
	\\ \\ 
	\textbf{Mark: 8/10}
	
%\subsection{Exception}
%	*note getUse(username)r is presumed to have be renamed getUserDetails(username). getEmail(username) and getPhoneNumber(username). are also available.\\ \\
%	
%	\noindent Test: getUserDetails called on non-existing user.\\
%	Result: Returned string "the user does not exist".\\\\
%	Test: getEmail(username) called on non-existing user.\\
%	Result: Returned string "the user does not exist".\\\\
%	Test: getPhoneNumber(username) called on non-existing user.\\
%	Result: Returned string "the user does not exist".\\
%	
%	\noindent getUser(username) substitutes function as expected under the conditions stipulated.
%	8/10
%	
%\subsection{Postcondition}
%	No postcondition is present, resulting in no change occurring, as expected.
%	10/10
%	
%\subsection{Return}
%	No function is evident in the code that returns a user object.
%	0/10
%	

\pagebreak
\section{Non-functional Requirements}
	Testing on non-functional requirements included:\\
%	\indent Scalability
%	\indent Efficiency
%	\indent Batch processing
%	\indent Duplicity prevention
	
	\begin{itemize}
		\item Scalability.
		\item Efficiency.
		\item Batch Processing.
		\item Duplicity Prevention
	\end{itemize}
	
	We have added 1000 dummy users were added to more accurately test the code.
	
\subsection{Scalability}
	PostgreSQL 9.5 can achieve about 400 000 TPS in select-only pgbench test. Considering that the system will likely receive roughly 30 000 users per year for the foreseeable 		future, it is unlikely that there would be any large impact on performance in the future beyond the annoyance of having to add each user individually. Scalability sits at an expected level 		without offering anything remarkable. \\ \\
	\textbf{Mark: 9/10}
	
\subsection{Efficiency} 
	Added 1000 users without much delay, testing with enough users to tax the system is outside of the scope of our ability to test. \\ \\ 
	\textbf{Mark: 10/10} 
	
\subsection{Batch Processing}
	Batch processing is not within the capabilities of the Gladius-Users system.\\ \\ 
	\textbf{Mark: 0/10}
	
\subsection{Duplicity Prevention}
	0 out of 1000 test users could be added multiple times. \\ \\ 
	\textbf{Mark: 10/10}

\end{document}
